\anonsection{Введение}

В любой достаточно востребованной современной системе генерируется и хранится огромное количество данных. Кроме того, почти всегда необходимо иметь возможность быстро получать доступ к необходимой в данный момент информации. Чтобы обеспечить подобную функциональность, необходимо использование различных алгоритмов и структур данных. 

Структуры данных, предназначенные для выполнения поисковых запросов, обычно назваются поисковыми индексами, а объекты, по которым осуществляется поиск~--- документами. Область научного знания, занимающаяся вопросами эффективного поиска называется Information Retrieval (IR).

Особенно важным случаем является так называемый полнотекстовый поиск, когда документы содержат текст на естественном языке. Естественный язык~--- используемый человеком для общения, например русский или английский. Важным этот случай является из-за большого количество задач, в которых он встречается. Поиск на форуме, в почтовом ящике, в интернете. Все эти задачи требуют реализации полнотекстового поиска.

Помимо всего прочего, большинство существующих систем, взаимодействующих с пользователем, постоянно пополняются данными. Ежесекундно приходят миллионы писем, публикуются тысячи статей и создаются сотни станиц в интернете. Для того, чтобы индексы содержали актуальные данные, необходим непрерывный процесс индексации~--- переноса новой информации в индекс.

В данной работе описывается процесс создания системы инструментов для внедрения полнотекстового поиска с налаженным процессом индексации в конечное приложение.

\clearpage
