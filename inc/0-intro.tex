\anonsection{Введение}

В любой современной системе, которая обладает достаточной востребованностью, генерируется и хранится огромное количество данных, которые требуют упорядочивания и позволяют быстро получить доступ к необходимой информации по определенным критериям. Чтобы обеспечить такую функциональность, при условии больших объемов данных, необходимо поддерживать вспомогательные структуры данных, так называемые индексы, которые решают задачу поиска документов, цельных единиц информации, по их свойствам.

Знание того, как построить индекс для конкретных документов, их свойств и доменов, которым эти свойства принадлежат, включается в отдельную научную область, которая называется Information Retrieval (IR). Существует множество научных работ, посвященным математическим моделям, соответствующим различным стратегиям поиска информации и, соотвественно, различным структурам данных, необходимым для эффективной реализации данных стратегий.

Среди всех прочих случаев поиска выделяется так называемый полнотекстовый поиск, когда домен некоторых свойств документов содержит в себе множество текстов на естественных языках, то есть используемых человеком для общения с другими людьми. Данный случай, за счет популярности задач, в которых возникает такая необходимость, рассмотрен научным сообществом особенно пристально и обладает специфичными подходами к поиску данных.

Помимо всего прочего, большинство существующих систем, взаимодействующих с пользователем, постоянно пополняются данными, поэтому необходим непрерывный процесс доставки новых данных в индекс, индексации, для возможности поиска по свежим данным.

В данной работе решена задача создания системы инструментов для внедрения полнотекстового поиска с налаженным процессом индексации в конечное приложение.

\clearpage
