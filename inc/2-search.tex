\section{Полнотекстовый поиск}

\subsection{Введение}

Оговоримся заранее, что интересующие нас данные представлены в виде \textbf{документов}, каждый из которых имеет набор \textbf{полей}, некоторых свойств, значения которых принадлежат некоторому домену. Для удобства легче всего считать, что все поля~--- байтовые строки. Это представление удобно тем, что естественно для данных, представимых внутри компьютера и кроме того, лексикографический порядок байтовых массивов соотвествует естественному порядку чисел, дат и лексикографическому порядку строк в представлении UTF-16. Также удобно считать, что одному и тому же полю может соотвествовать несколько значений. Пара (поле, значение) называется \textbf{термом} и обладает семантикой слова в тексте.

Любой документ можно привести к такой форме. Даты например при этом переводятся в числа, а числа в байтовые строки фиксированной длины, соответсвующей размерности числа. Процесс приведения текста чуть более сложен и включет в себя две стадии~--- токенизирование, когда текст превращаеся в набор слов-токенов, каждый из которых затем анализируется, то есть приводится к нормальной форме. Для естественных языков это может быть например выделение корня из слова. Морфологический анализ слов является отдельной задачей, которая используется при построени полнотекстовых индексов, но которую мы не будем подробно затрагивать далее в тексте.

После того, как каждый документ приведен к форме набора термов, можно составить так называемые прямой и обратный индексы. Прямой индекс, который мы можем во многих формах наблюдать в реальности, отображает документ в виде какого-то уникального индитификатора, например абстрактного номера, в набор термов. Обратный индекс нужен для противоположной задачи, отобразить терм в набор документов, в которых он встречается. После того, как мы построили эти индексы, мы может обрабатывать запросы, включающие в себя термы, их отрицание и логические связки. 

\clearpage
