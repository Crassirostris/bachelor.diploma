\anonsection{Заключение}

В результате работы была проанализирована задача создания системы для построения полнотекстовых поисковых индексов, поставлены требования по функциональным возможностям, надежности и удобству использования, были сделаны выводы о необходимости привлечения внешнего решения для обеспечения одной части задачи, ровно как необходимость создания собственного продукта для аппеляции другой части.

Были проанализированы задачи и возможные готовые решения проблемы построения и управления полнотекстовыми индексами в распределенной системе с большим количеством данных, сформулированы критерии для выбора наиболее подходящего продукта и на основе этих требований и сторонних исследований был сделан выбор в стороны одного из предложенных кандидатов.

Были проанализированы требования к новому решению по непрерывной подаче данных из первичного источника в индекс и сформирован список критериев, согласно которым необходимо было спроектировать и разработать распределенную систему запуска индексаторов, после чего данная система была реализована, и ее корректность и надежность были проверены тщательным анализом алгоритма обработки нестандартных ситуаций.

Дальнейшее развитие данной работы видится в повышении надежности сохранности данных путем введения механизмов отката процесса индексации и введением большего количества критериев для мониторинга системы. Также одним из направлений будущего развития может стать развитие пользовательского интерфейса и процессов введение в эксплуатацию новых индексаторов для достижения большей степени удобства и больше степени автоматизации в процессе эксплуатации и обновления различных компонентов проекта.

\clearpage
