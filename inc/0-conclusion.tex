\anonsection{Заключение}

В рамках данной работы было проанализировано множество задач, которые необходимо решить для успешного создания системы, удовлетворяющей выставленным требованиям. В результате исходная задача была декомпозирована до двух независимых компонентов, каждая из которых была независимо решена.

Был сделан вывод о необходимости использования готового решения для создания и управления распределенными поисковыми индексами, проведено сравнение существующих доступных вариантов и выбран наиболее подходящий с точки зрения функциональности и производительности.

Анализ существующих решений для индексации данных показал, что подходящего под все требования готового решения не существует, или не известно автору. По этой причине было решено создать собственный продукт для обеспечения полнотекстовых индексов непрерывным процессом индексации. Был выдвинут список требований и в соответствии с ним реализован программный продукт.

На данный момент система успешно внедрена и эксплуатируется более, чем 20 приложениями одновременно. В различных индексах расположено уже более 18 ТБ данных, к которым суммарно ежедневно обращаются более миллиона раз.

Дальнейшее развитие данной работы видится в повышении надежности сохранности данных путем введения механизмов отката процесса индексации и введением большего количества критериев для мониторинга системы. Также одним из направлений будущего развития может стать развитие пользовательского интерфейса и процессов введение в эксплуатацию новых индексаторов для достижения большей степени удобства и больше степени автоматизации в процессе эксплуатации и обновления различных компонентов проекта.

\clearpage
