\section{Постановка задачи}

На момент постановки задачи не существовало известного автору универсального, масштабируемого и отказоустойчивого решения для внедрения полнотекстового поиска с налаженным процессом индексации в существующее приложение, не существует и до сих пор. В результате работы хотелось насколько это возможно приблизится к появлению такого продукта.

Первая проблема касается почти всех существующих решений и состоит в том, что настройка и обслуживание требует достаточно много ручной работы. Например, популярный набор инструментов ELK требует самостоятельной настройки и отслеживания работоспособности каждого индексатора. Будучи весьма удобным в использовании для одного приложения, данный инструмент порождает множество проблем при масштабировании в 10, 100 раз.

Вторая проблема заключается в отсутствии у многих решений способности автоматически горизонтально масштабировать индексы. Отсутствие масштабирования ограничивает сверху размер поисковых структур и запасы по производительности при неизменных доступных вычислительных мощностях. Индекс размером в 20ТБ проблематично даже физически разместить на одном сервере, не говоря уж о манипуляциях над таким объемом данных. В это же время кластерные решения обрабатывают и большие структуры.

Третья проблема, свойственная некоторым из существующих решений, состоит в навязывании определенной платформы для разработки конечного продукта. Однако задача часто состоит во внедрении полнотекстового поиска в уже существующее приложение и навязывание платформы недопустимо.

Таким образом, необходимо было создать решение, которое обладало бы большой степенью автоматизации, хорошо масштабировалось бы на большие объемы данных и большое количество использований и было бы дружелюбно к приложению на любой платформе. При этом получившаяся система должна была быть отказоустойчивой и достаточно производительной.

\clearpage
